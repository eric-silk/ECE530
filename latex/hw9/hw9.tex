\documentclass[11pt]{report}
\usepackage{amsmath}
\usepackage{amssymb}
\usepackage{amsfonts}
\usepackage{mathtools}
\usepackage{amsthm}
\usepackage{ragged2e}
\usepackage[hidelinks]{hyperref}
\usepackage{float}
\usepackage{pgf,tikz}
\usepackage[shortlabels]{enumitem}
\usepackage{color}
\usepackage{pgfplots}
\usepackage[margin = 1 in]{geometry}
\usepackage{mathrsfs}
\usetikzlibrary{arrows}
\usepackage{multicol}
\usepackage{fancyhdr}
\pagestyle{fancy}
\usepackage{multirow}
\usepackage{graphicx}
\usepackage{psfrag}
\usepackage{biblatex}
\addbibresource{hw8.bib}
\usepackage{listings}
\usepackage{algorithm}
\usepackage{algorithmic}
\renewcommand{\footrulewidth}{0.4pt}

\newtheorem{theorem}{Theorem}[chapter]
\newtheorem{defn}{Definition}[chapter]
\newtheorem{lemma}{Lemma}[chapter]

\theoremstyle{definition}
\newtheorem{proposition}{Proposition}[chapter]
\newtheorem{remark}{Remark}[chapter]
\newtheorem{example}{Example}[chapter]

\DeclareMathOperator*{\argmin}{arg\,min}
\DeclareMathOperator*{\argmax}{arg\,max}

\newcommand{\user}{}
\newcommand{\xlr}[2]{#1 \left(#2\right)}
\newcommand{\clr}[2]{#1 \left\{ #2 \right\}}
\newcommand{\rank}{\mathrm{rank}}
\newcommand{\mat}[1]{\mathbf{#1}}
\lhead{ECE 530 - Fall 2023 at University of Illinois at Urbana-Champaign}
\rhead{HW9}
\lfoot{Author: \textcolor{red}{Eric Silk, esilk2}}
\rfoot{Due: Fri. December 15}
\begin{document}


\section*{Problem 1: Where do you belong, Krylov?}
\subsection*{Solution}

\subsubsection*{a}
\[
	Aq^1
	= A\frac{b}{\|b\|}
	\in \mathrm{span}(\left\{b, Ab\right\})
	= \mathcal{K}_2
	= \mathrm{span}(\left\{q^1, Aq^1\right\})
	= \mathrm{span}(\left\{q^1, q^2\right\})
\]
\[ v^{i+1} = Aq^i - \sum_{k=1}^{i}\langle Aq^i,q^k\rangle q^k \]
Note that the inner products are scalars and denote them as $\alpha_k$:
\[  = Aq^i - \sum_{k=1}^{i}\alpha_k q^k \]
\[q^{i+1} = \frac{v^{i+1}}{\|v^{i+1}\|} = \frac{Aq^i - \sum_{k=1}^{i}\alpha_kq^k}{\|Aq^i - \sum_{k=1}^{i}\alpha_kq^k\|}\]
Again, note that the denominator norm is a scalar; we will call it $\gamma_i$. Substitute and re-arrange:
\[ Aq^i = \gamma_i q^{i+1} + \sum_{k=1}^{i}\alpha_kq^k \]
\[\implies Aq^{i} \in \mathrm{span}(\left\{q_1,\ldots, q^{i+1}\right\}) = \mathcal{K}_{i+1}\]
Then, we note that for $i\geq m$, $\mathcal{K}_{i} = \mathcal{K}_{m}$. So:
\[ j = \min(i+1, m) \]

\subsubsection*{b}
Taken from \textit{Numerical Linear Algebra} by Trefethen and Bau.

Consider the Hessenberg decomposition:
\[A = PHP^*\]
where $P$ is a unitary matrix (i.e. $PP^*=P^*P=PP^{-1}=I$). Note that orthonormal matrices are unitary,
and restrict our case to state we wish to perform a Hessenberg decomposition via orthonormal matrix $Q$. Then:
\[A = QHQ^* \implies AQ = QH\]
Let $Q_n$ be the first $n$ columns of $Q$ and $H_n$ to be the $(n+1)\times n$ upper-left section of $H$ (which
is also Hessenberg!). Then:
\[AQ_n = Q_{n+1}H_n\]
The $n$-th column of the equation is:
\[Aq_n = h1_nq_1+\ldots+h_{nn}q_n+h_{n+1,n}q_{n+1}\]
Solving for $q_{n+1}$ will require a technique to ensure it's orthonormality -- if we use Modified Gram-Schmidt,
we get:
\begin{algorithm}[H]
	\caption{Algorithm to solve the above equation with an orthonormal $q_{n+1}$}
	\begin{algorithmic}
		\STATE $b \in \mathbb{R}^{n}$, $q_1 \gets \frac{b}{\|b\|}$
		\FOR{$n=1,2,3\ldots$}
		\STATE $v\gets Aq_n$
		\FOR{$j=1,\ldots,n$}
		\STATE $h_{jn}\gets q^*_jv$
		\STATE $v \gets v - h_{jn}q_{j}$
		\ENDFOR
		\STATE $h_{n+1,n}\gets \|v\|$
		\STATE $q_{n+1}\gets \frac{v}{h_{n+1,n}}$
		\ENDFOR
	\end{algorithmic}
\end{algorithm}
...which is identical to the provided iterate equations.\qed



\subsubsection*{c}
See the code section at the end for the implementation. The result is:

\begin{lstlisting}[basicstyle=\small]
Did arnoldi work:
True
Q:
[[0.58 0.31 -0.75]
	[-0.58 0.81 -0.11]
	[0.58 0.50 0.65]]
H:
[[0.33 3.09 0.00]
	[3.09 7.56 0.95]
	[0.00 0.95 -0.90]]
Wow its upper Hessenberg, whoda thunk it
Wow its lower Hessenberg, whoda thunk it
Wow it's tridiagonal, its cuz the matrix A is Hermitian (or Symmetric cuz its real)
\end{lstlisting}

\newpage

\section*{Problem 2: Ask Gram, Schmidt, or Givens for QR}

\subsection*{Solution}
\subsubsection*{a}
Taken from ``Linear Algebra Done Right'' by Sheldon Axler.

Suppose $v_1,\ldots,v_m$ is a lin. ind. list of vectors in $V$. Let $e_1=\frac{v_1}{\|v_1\|}$. For
$j=2,\ldots,m$, define $e_j$ inductively by:
\[
	e_j = \frac{
	v_j-\langle v_j,e_1\rangle e_1-\cdots -\langle v_j,e_{j-1}\rangle e_{j-1}
	}{
	\|v_j-\langle v_j,e_1\rangle e_1-\cdots -\langle v_j,e_{j-1}\rangle e_{j-1}\|
	}
\]
(i.e. the Gram-Schmidt process definition). Then, $e_1,\ldots,e_m$ is an orthonormal list of vectors in $V$ s.t.:
\[\mathrm{span}(v_1,\ldots,v_j)=\mathrm{span}(e_1,\ldots,e_j)\forall j=1,\ldots,m\]


Note that for $j=1$, $\mathrm{span}(v_1)=\mathrm{span}(e_1)$ because $v_1$ is a positive mulitple of $e_1$.

Suppose $1<j<m$ and it has been verified that:
\[\mathrm{span}(v_1,\ldots,v_{j-1})=\mathrm{span}(e_1,\ldots,e_{j-1})\]
Note that $v_j\notin \mathrm{span}(v_1,\ldots,v_{j-1})$ because $v_1\ldots,v_m$ are linearly independent.
Thus, $v_j\notin \mathrm{span}(e_1,\ldots,e_{j-1})$. As such, we are not dividing by zero in the definition of $e_j$.
We can also see that $\|e_j\|=1$ by its definition.

Let $k\in[1,j)$. Then:
\[
	\langle e_j,e_k\rangle =
	\bigg\langle
	\frac{v_j-\langle v_j,e_1\rangle e_1 -\ldots - \langle v_j, e_{j-1}\rangle e_{j-1}}
	{v_j-\langle v_j,e_1\rangle e_1 -\ldots - \langle v_j, e_{j-1}\rangle e_{j-1}}
	, e_k
	\bigg\rangle
\]
\[
	= \frac{\langle v_j,e_k\rangle - \langle v_j, e_k\rangle}
	{v_j-\langle v_j,e_1\rangle e_1 -\ldots - \langle v_j, e_{j-1}\rangle e_{j-1}}
\]
\[ = 0\]
Thus $e_1,\ldots,e_j$ is an orthonormal list.

From the definition of $e_j$, we see that $v_j\in\mathrm{span}(e_1,\ldots,e_j)$. Combined with the
equivalency of the spans provided above, we know:
\[\mathrm{span}(v_1,\ldots,v_j)\subset\mathrm{span}(e_1,\ldots,e_j)\]
Both these lists are lin. ind., thus both subspaces have dimension $j$ and are equal.\qed

\subsubsection*{b}

\begin{algorithm}
	\caption{Classical Gram-Schmidt, annotated with FLOP counts}
	\begin{algorithmic}
		\FOR{$j=1,\ldots,n$}
		\STATE $v\gets a^j$
		\FOR{$i=1,\ldots,j-1$}
		\STATE $R_{ij}\gets\langle v,q^i\rangle$ \COMMENT{$m$ multiplications, $(m-1)$ additions $\implies 2m-1$}
		\STATE $v\gets v-R_{ij}q^i$ \COMMENT{$m$ multiplications, $m$ subtractions $\implies 2m$}
		\ENDFOR \COMMENT{Total cost is }
		\STATE $q^j\gets \frac{v}{\|v\|}$ \COMMENT{$m$ multiplications, $m-1$ additions, $1$ for sqrt $\implies 2m$}
		\STATE $R_{jj}\gets \langle a^j,q^j\rangle$ \COMMENT{$m$ multiplications, $(m-1)$ additions $\implies 2m-1$}
		\ENDFOR
		\STATE $Q = (q^1|\ldots|q^n)$
	\end{algorithmic}
\end{algorithm}
The inner loop requires:
\[(4m-1)(1-1)+(4m-1)+(2-1)+\ldots+(4m-1)(n-1) = (4m-1)(0+1+\ldots+n-1)=(4m-1)(\frac{1}{2}n(n+1)-1)\]
The outer portion requires $(4m-1)*n$ operations. Summing:
\[=(4m+1)(\frac{1}{2}n(n+1)-1)+(4m+1)(n)=(4m+1)(\frac{1}{2}n(n)-1+n)\]
Simplifying and discarding lower order terms we find:
\[=2mn^2\]

\subsubsection*{c}
An upper Hessenberg matrix of size $n\times n$ will require $n-1$ operations to zero out the
non-zero elements below the diagonal, which will result in a QR decomposition.
Givens matrices take the form:
\[
	G_n = \begin{bmatrix}
		1      & \cdots & 0      & \cdots & 0      & \cdots & 0      \\
		\vdots & \ddots & \vdots &        & \vdots &        & \vdots \\
		0      & \cdots & c      & \cdots & -s     & \cdots & 0      \\
		\vdots &        & \vdots & \ddots & \vdots &        & \vdots \\
		0      & \cdots & s      & \cdots & c      & \cdots & \vdots \\
		\vdots &        & \vdots &        & \vdots & \ddots & \vdots \\
		0      & \cdots & 0      & \cdots & 0      & \cdots & 1
	\end{bmatrix}
\]
The location of the $c$ and $s$ terms correspond to zeroing element $i$ and $j$;
that is, to zero element $A_{i,j}$, we set $s=G_{ji}=-G{ij}$ and $G_{k,k}=c$ for
$k=i,j$ (and 1 for $k\neq i,j)$. The resulting algorithm is as follows:
\begin{algorithm}
	\caption{QR Factorization of an Upper Hessenberg Matrix using Givens rotations}
	\begin{algorithmic}
		\STATE Input: $A^1\in\mathbb{R}^{n\times n}$, $A_{ij}=0\forall i>j+1$
		\COMMENT{This could be a parallelized ``loop''}
		\FOR{$j=1,\ldots,n-1$}
		\STATE $i\gets j+1$
		\STATE $r^{j}\gets \sqrt{\left[{A}^{j}_{jj}\right]^2+\left[A^{j}_{ij}\right]^2}$
		\STATE $c\gets A^j_{jj}/r$
		\STATE $s \gets -A^j_{ij}/r$
		\STATE $G^{j}_{ji} \gets s$
		\STATE $G^{j}_{ij}\gets -s$
		\STATE $G^{j}_{ii}\gets c$
		\STATE $G^{j}_{jj}\gets c$
		\STATE $A^{j+1} \gets G^{j}A^j$
		\ENDFOR
		\STATE $Q = \Pi_{j=1}^{n-1} G^{j}$
		\STATE $R = A^{n-1}$
	\end{algorithmic}
\end{algorithm}


\newpage
\section*{Code}
\definecolor{codegreen}{rgb}{0,0.6,0}
\definecolor{codegray}{rgb}{0.5,0.5,0.5}
\definecolor{codepurple}{rgb}{0.58,0,0.82}
\definecolor{backcolour}{rgb}{0.95,0.95,0.92}
\lstdefinestyle{mystyle}{
	backgroundcolor=\color{backcolour},
	commentstyle=\color{codegreen},
	keywordstyle=\color{magenta},
	numberstyle=\tiny\color{codegray},
	stringstyle=\color{codepurple},
	basicstyle=\ttfamily\footnotesize,
	breakatwhitespace=false,
	breaklines=true,
	captionpos=b,
	keepspaces=true,
	numbers=left,
	numbersep=5pt,
	showspaces=false,
	showstringspaces=false,
	showtabs=false,
	tabsize=2
}
\lstset{style=mystyle}


\subsection*{Problem 1}
\lstinputlisting[
	language=Python,
	basicstyle=\tiny
]{../../ece530/ece530/hw9/arnoldi.py}
\end{document}
