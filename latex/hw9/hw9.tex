\documentclass[11pt]{report}
\usepackage{amsmath}
\usepackage{amssymb}
\usepackage{amsfonts}
\usepackage{mathtools}
\usepackage{amsthm}
\usepackage{ragged2e}
\usepackage[hidelinks]{hyperref}
\usepackage{float}
\usepackage{pgf,tikz}
\usepackage[shortlabels]{enumitem}
\usepackage{color}
\usepackage{pgfplots}
\usepackage[margin = 1 in]{geometry}
\usepackage{mathrsfs}
\usetikzlibrary{arrows}
\usepackage{multicol}
\usepackage{fancyhdr}
\pagestyle{fancy}
\usepackage{multirow}
\usepackage{graphicx}
\usepackage{psfrag}
\usepackage{biblatex}
\addbibresource{hw8.bib}
\usepackage{listings}
\usepackage{algorithmic}
\renewcommand{\footrulewidth}{0.4pt}

\newtheorem{theorem}{Theorem}[chapter]
\newtheorem{defn}{Definition}[chapter]
\newtheorem{lemma}{Lemma}[chapter]

\theoremstyle{definition}
\newtheorem{proposition}{Proposition}[chapter]
\newtheorem{remark}{Remark}[chapter]
\newtheorem{example}{Example}[chapter]

\DeclareMathOperator*{\argmin}{arg\,min}
\DeclareMathOperator*{\argmax}{arg\,max}

\newcommand{\user}{}
\newcommand{\xlr}[2]{#1 \left(#2\right)}
\newcommand{\clr}[2]{#1 \left\{ #2 \right\}}
\newcommand{\rank}{\mathrm{rank}}
\newcommand{\mat}[1]{\mathbf{#1}}
\lhead{ECE 530 - Fall 2023 at University of Illinois at Urbana-Champaign}
\rhead{HW9}
\lfoot{Author: \textcolor{red}{Eric Silk, esilk2}}
\rfoot{Due: Fri. December 15}
\begin{document}


\section*{Problem 1: Where do you belong, Krylov?}
\subsection*{Problem Statement}

\subsubsection*{a}

\subsubsection*{b}

\subsubsection*{c}

\subsection*{Solution}

\subsubsection*{a}

\[
	Aq^i = \frac{q^{i+1}}{\|q^{i+1}\|} + \sum_{k}^{i}\langle Aq^i,q^k\rangle q^k \in \mathcal{K}_{i+1}
\]

\subsubsection*{b}
\[
	\begin{bmatrix}
		Aq_1 & | & \ldots & | & Aq_n
	\end{bmatrix}
	=
	\begin{bmatrix}
		Qh_1 & | & \ldots & | & Qh_n
	\end{bmatrix}
\]
\[\implies Aq_1=Qh_1\]
\[Aq_1\in\mathcal{K}_2\implies Qh_1\in\mathcal{K}_2\]
\[
	Aq_1 = \begin{bmatrix}
		* \\ * \\ 0 \\ 0 \\ \vdots \\ 0
	\end{bmatrix},\
	Aq_2 = \begin{bmatrix}
		* \\ * \\ * \\ 0 \\ \vdots \\ 0
	\end{bmatrix},\ \ldots
	Aq_n = \begin{bmatrix}
		* \\ * \\ * \\ * \\ \vdots \\ *
	\end{bmatrix}
\]

\subsubsection*{c}

\newpage

\section*{Problem 2: Ask Gram, Schmidt, or Givens for QR}
\subsection*{Problem Statement}
\subsubsection*{a}

\subsubsection*{b}

\subsubsection*{c}

\subsection*{Solution}
\subsubsection*{a}
Taken from ``Linear Algebra Done Right'' by Sheldon Axler.

Suppose $v_1,\ldots,v_m$ is a lin. ind. list of vectors in $V$. Let $e_1=\frac{v_1}{\|v_1\|}$. For
$j=2,\ldots,m$, define $e_j$ inductively by:
\[
	e_j = \frac{
	v_j-\langle v_j,e_1\rangle e_1-\cdots -\langle v_j,e_{j-1}\rangle e_{j-1}
	}{
	\|v_j-\langle v_j,e_1\rangle e_1-\cdots -\langle v_j,e_{j-1}\rangle e_{j-1}\|
	}
\]
(i.e. the Gram-Schmidt process definition). Then, $e_1,\ldots,e_m$ is an orthonormal list of vectors in $V$ s.t.:
\[\mathrm{span}(v_1,\ldots,v_j)=\mathrm{span}(e_1,\ldots,e_j)\forall j=1,\ldots,m\]


Note that for $j=1$, $\mathrm{span}(v_1)=\mathrm{span}(e_1)$ because $v_1$ is a positive mulitple of $e_1$.

Suppose $1<j<m$ and it has been verified that:
\[\mathrm{span}(v_1,\ldots,v_{j-1})=\mathrm{span}(e_1,\ldots,e_{j-1})\]
Note that $v_j\notin \mathrm{span}(v_1,\ldots,v_{j-1})$ because $v_1\ldots,v_m$ are linearly independent.
Thus, $v_j\notin \mathrm{span}(e_1,\ldots,e_{j-1})$. As such, we are not dividing by zero in the definition of $e_j$.
We can also see that $\|e_j\|=1$ by its definition.

Let $k\in[1,j)$. Then:
\[
	\langle e_j,e_k\rangle =
	\bigg\langle
	\frac{v_j-\langle v_j,e_1\rangle e_1 -\ldots - \langle v_j, e_{j-1}\rangle e_{j-1}}
	{v_j-\langle v_j,e_1\rangle e_1 -\ldots - \langle v_j, e_{j-1}\rangle e_{j-1}}
	, e_k
	\bigg\rangle
\]
\[
	= \frac{\langle v_j,e_k\rangle - \langle v_j, e_k\rangle}
	{v_j-\langle v_j,e_1\rangle e_1 -\ldots - \langle v_j, e_{j-1}\rangle e_{j-1}}
\]
\[ = 0\]
Thus $e_1,\ldots,e_j$ is an orthonormal list.

From the definition of $e_j$, we see that $v_j\in\mathrm{span}(e_1,\ldots,e_j)$. Combined with the
equivalency of the spans provided above, we know:
\[\mathrm{span}(v_1,\ldots,v_j)\subset\mathrm{span}(e_1,\ldots,e_j)\]
Both these lists are lin. ind., thus both subspaces have dimension $j$ and are equal.\qed

\subsubsection*{b}
\begin{algorithmic}
	\FOR{$j=1,\ldots,n$}
	\STATE $v\gets a^j$
	\FOR{$i=1,\ldots,j-1$}
	\STATE $R_{ij}\gets\langle v,q^i\rangle$ \COMMENT{$m$ multiplications, $(m-1)$ additions $\implies 2m-1$}
	\STATE $v\gets v-R_{ij}q^i$ \COMMENT{$m$ multiplications, $m$ subtractions $\implies 2m$}
	\ENDFOR \COMMENT{Total cost is }
	\STATE $q^j\gets \frac{v}{\|v\|}$ \COMMENT{$m$ multiplications, $m-1$ additions, $1$ for sqrt $\implies 2m$}
	\STATE $R_{jj}\gets \langle a^j,q^j\rangle$ \COMMENT{$m$ multiplications, $(m-1)$ additions $\implies 2m-1$}
	\ENDFOR
	\STATE $Q = (q^1|\ldots|q^n)$
\end{algorithmic}
The inner loop requires:
\[(4m-1)(1-1)+(4m-1)+(2-1)+\ldots+(4m-1)(n-1) = (4m-1)(0+1+\ldots+n-1)=(4m-1)(\frac{1}{2}n(n+1)-1)\]
The outer portion requires $(4m-1)*n$ operations. Summing:
\[=(4m+1)(\frac{1}{2}n(n+1)-1)+(4m+1)(n)=(4m+1)(\frac{1}{2}n(n)-1+n)\]
Simplifying and discarding lower order terms we find:
\[=2mn^2\]

\subsubsection*{c}
An upper Hessenberg matrix of size $n\times n$ will require $n-1$ operations to zero out the
non-zero elements below the diagonal, which will result in a QR decomposition.
Givens matrices take the form:
\[
	G_n = \begin{bmatrix}
		1      & \cdots & 0      & \cdots & 0      & \cdots & 0      \\
		\vdots & \ddots & \vdots &        & \vdots &        & \vdots \\
		0      & \cdots & \vdots &        & \vdots &        & \vdots \\
		\vdots &        & \vdots &        & \vdots &        & \vdots \\
		0      & \cdots & \vdots &        & \vdots &        & \vdots \\
		\vdots &        & \vdots &        & \vdots &        & \vdots \\
		0      & \cdots & 0      & \cdots & 0      & \cdots & 1
	\end{bmatrix}
\]


\newpage
\section*{Code}
\definecolor{codegreen}{rgb}{0,0.6,0}
\definecolor{codegray}{rgb}{0.5,0.5,0.5}
\definecolor{codepurple}{rgb}{0.58,0,0.82}
\definecolor{backcolour}{rgb}{0.95,0.95,0.92}
\lstdefinestyle{mystyle}{
	backgroundcolor=\color{backcolour},
	commentstyle=\color{codegreen},
	keywordstyle=\color{magenta},
	numberstyle=\tiny\color{codegray},
	stringstyle=\color{codepurple},
	basicstyle=\ttfamily\footnotesize,
	breakatwhitespace=false,
	breaklines=true,
	captionpos=b,
	keepspaces=true,
	numbers=left,
	numbersep=5pt,
	showspaces=false,
	showstringspaces=false,
	showtabs=false,
	tabsize=2
}
\lstset{style=mystyle}


\subsection*{Problem 1}
\lstinputlisting[
	language=Python,
	basicstyle=\tiny
]{../../ece530/ece530/hw9/arnoldi.py}
\end{document}
