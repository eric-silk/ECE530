\documentclass[11pt]{report}
\usepackage{amsmath}
\usepackage{amssymb}
\usepackage{amsfonts}
\usepackage{mathtools}
\usepackage{amsthm}
\usepackage{ragged2e}
\usepackage[hidelinks]{hyperref}
\usepackage{float}
\usepackage{pgf,tikz}
\usepackage[shortlabels]{enumitem}
\usepackage{color}
\usepackage{pgfplots}
\usepackage[margin = 1 in]{geometry}
\usepackage{mathrsfs}
\usetikzlibrary{arrows}
\usepackage{multicol}
\usepackage{fancyhdr}
\pagestyle{fancy}
\usepackage{multirow}
\usepackage{graphicx}
\usepackage{psfrag}
\usepackage{biblatex}
\addbibresource{hw7.bib}
\usepackage{listings}
\renewcommand{\footrulewidth}{0.4pt}

\newtheorem{theorem}{Theorem}[chapter]
\newtheorem{defn}{Definition}[chapter]
\newtheorem{lemma}{Lemma}[chapter]

\theoremstyle{definition}
\newtheorem{proposition}{Proposition}[chapter]
\newtheorem{remark}{Remark}[chapter]
\newtheorem{example}{Example}[chapter]

\DeclareMathOperator*{\argmin}{arg\,min}
\DeclareMathOperator*{\argmax}{arg\,max}

\newcommand{\user}{}
\newcommand{\xlr}[2]{#1 \left(#2\right)}
\newcommand{\clr}[2]{#1 \left\{ #2 \right\}}
\newcommand{\rank}{\mathrm{rank}}
\newcommand{\mat}[1]{\mathbf{#1}}
\lhead{ECE 530 - Fall 2023 at University of Illinois at Urbana-Champaign}
\rhead{HW7}
\lfoot{Author: \textcolor{red}{Eric Silk, esilk2}}
\rfoot{Due: Wed. November 8}
\begin{document}


\section*{Problem 1: Determinants can be floppy}
\subsection*{Problem Statement}
Consider a non-singular matrix $\mat{A}\in\mathbb{R}^{n\times n}$, given by
\begin{equation}
	\mat{A} = \begin{pmatrix}
		a_{11} & a_{12} & \cdots & a_{1n} \\
		a_{21} & a_{22} & \cdots & a_{2n} \\
		\vdots & \vdots & \ddots & \vdots \\
		a_{n1} & a_{n2} & \cdots & a_{nn}
	\end{pmatrix}
\end{equation}
Let us compute its determinant using the following formula:
\[\det(\mat{A}) \coloneqq \sum_{k=1}^{n}(-1)^ka_{1k}\det(\widehat{\mat{A}}^{1k})\]
where $\widehat{\mat{A}}^{1k}\in\mathbb{R}^{(n-1)\times(n-1)}$ is the matrix
obtained by removing the first row and $k$-th column of $\mat{A}$.

\subsubsection*{a}
Suppose for $\mat{A}\in\mathbb{R}^{n\times n}$, computing $\det(\mat{A})$ using the above
formula takes $y_n$ flops. Find the relationship between $y_n$ and $y_{n-1}$. Assume that
negating a number does not require  a flop.

\subsubsection*{b}
Use your recurrence relation in (a) to prove that $n!\leq y_n\leq \frac{1}{2}(n+1)!$ for
each $n\geq 5$. Here, $z!=z(z-1)(z-2)\ldots1$ denotes the factorial of $z$.

\subsubsection*{c}
Estimate how much time your computer takes for one flop. Please submit your code.

\subsubsection*{d}
\begin{enumerate}
	\item Using your answer in (c), estimate how long it will take to compute
	      $\det(\mat{A})$ for the provided equation, using $n=100$. Answer in number
	      of years! The following Sterling's approximation may be useful:
	      \[n! \approx \sqrt{2\pi n}\left(\frac{n}{e}\right)^n\]
	\item Based on your asnwer in (i), will you use this method to compute the
	      determinant $100\times 100$ matrix? By the way, the age of the universe is
	      13.8 billion years.

\end{enumerate}

\subsection*{Solution}
\subsubsection*{a}
Let $F(n)\coloneqq y_n$, i.e. the FLOP count of this calculation as a function
of $n$.  We can see that, for a given sub-term in the summation we have:
\[(-1)^ka_{1k}\det(\widehat{\mat{A}}^{1k})\]
The sub-matrix determinant will take $F(n-1)$ FLOPs. The negation will take
none. The multiplication of $a_{1k}$ and the sub-determinant will take one.
Each sub-term will thus take $F(n-1)+1$ FLOPS, and there are $n$ sub-terms.
Then, the summations will require $n-1$ additional flops. So, we find:
\[F(n) = n(F(n-1)+1)+(n-1) = nF(n-1)+2n-1\]
Note that $F(0)= F(1) = 0$, so:
\begin{table}[h]
	\begin{center}

		\begin{tabular}{l|lllllllll}
			$n$    & 1 & 2 & 3  & 4  & 5   & 6    & 7     & 8      & 9      \\ \hline
			$F(n)$ & 0 & 3 & 14 & 63 & 324 & 1955 & 13698 & 109599 & 986408
		\end{tabular}
	\end{center}
\end{table}

\subsubsection*{b}
\[ F(n) = nF(n-1)+2n-1 \]
\[ = n(n-1)F(n-2)+n(n-1) \]
\[ = n((n-1)((n-1)(\ldots)+n-2)+n-2)+n-1\]
\[ = 2n-1+n(2(n-1)-1)+n(n-1)(2(n-2)-1)+\ldots+n!\]
\[ < 2n+2n(n-1)+2n(n-1)(n-2)+\ldots2n!+n!\]
\[ = 2(n+n(n-1)+\ldots+n!/2)+3n!\]
\[ <2\left(\frac{n!}{(n-1)!}+\frac{n!}{(n-2)!}+\ldots+\frac{n!}{2!}\right)+3n!\]
Then, note that:
\[\frac{2n!}{k!}\leq \frac{n!}{(k-1)!}\forall k\geq 2\]
and thus:
\[\frac{n!}{(n-1)!}+\frac{n!}{(n-2)!}+\frac{n!}{(n-3)!}+\ldots+\frac{n!}{2!}\]
\[\leq \frac{n!}{(n-2)!}+\frac{n!}{(n-2)!}+\frac{n!}{(n-3)!}+\ldots+\frac{n!}{2!}\]
\[\leq \frac{n!}{(n-3)!}+\frac{n!}{(n-3)!}+\ldots+\frac{n!}{2!}\]
\[\leq \frac{n!}{(n-4)!}+\ldots+\frac{n!}{2!}\]
\[\ldots\]
\[\leq \frac{n!}{2!}+\frac{n!}{2!}\]
\[\leq n!\]
So:
\[
	F(n)
	= nF(n-1)+2n-1
	< 2\left(\frac{n!}{(n-1)!}+\frac{n!}{(n-2)!}+\ldots+\frac{n!}{2!}\right)+3n!
\]
\[\leq 2n!+3n! = 5n!\]
Furthermore, note that for $n\geq5$:
\[5n! \leq \frac{1}{2}(n+1)!\]
as

Referenced \cite{DeterminantStackOverflow} for deriviation.\footnote{There's
	something amusing about a recursive algorithm post on a website called
	``StackOverflow''...}


\subsection*{c}

\newpage

\section*{Problem 2: It Schur is Nice}
\subsection*{Problem Statement}
Now, let's compute the determinant of $\mat{A}\in\mathbb{R}^{n\times n}$ differently.
Write $\mat{A}$ as
\begin{equation}
	\mat{A} = \begin{pmatrix}
		a_{11}       & \mat{A}_{12} \\
		\mat{A}_{21} & \mat{A}_{22}
	\end{pmatrix}
\end{equation}

where $\mat{A}_{22}\in\mathbb{R}^{(n-1)\times(n-1)}$. Other blocks have
appropriate dimensions.  If $a_{11}\neq 0$, then
\[ \det(\mat{A})=a_{11}\det(\mat{A}_{22}-\mat{A}_{21}a_{11}^{-1}\mat{A}_{12}) \]
where $\mat{A}_{22}-\mat{A}_{21}a_{11}^{-1}\mat{A}_{12}$ is non-singular.
Suppose you can swap two rows without any FLOPs, but such a swap changes the
sign of the determinant.

\subsubsection*{a}
If computing $\det{A}$ using (2) takes $z_n$ flops, find a recurrence relation
between $z_n$ and $z_{n-1}$.

\subsubsection*{b}
Find the dominant term in $n$ in the expression of $z_n$ as a function of $n$;
i.e. without a recurrence relation.

\subsection*{c}
Will you use (2) to compute the determinant of a $100\times 100$ matrix? Answer
logically.


\subsection*{Solution}
\subsubsection*{a}
Applying the scalar divison to one of the matrices will require $n-1$
operations.The matrix-matrix multiplication is really the outer product of a
$(n-1)\times 1$ and $1\times(n-1)$ vector, requiring $(n-1)\times(n-1)=(n-1)^2$
operations. Finally, the subtraction of Matrices will require $(n-1)^2$ operations.
The result will be $n(n-1)^2+(n-1)^2=(n+1)(n-1)^2$ to produce an argument to the
next determinant, and the result must be multiplied once more.
\[ F(n) = F(n-1) + (n+1)(n-1)^2 + 1 \]

\subsubsection*{b}
Ignoring the $F(n-1)$ term, the resulting expression becomes:
\[F(n) = (n+1)(n-1)^2+1 = n^3-n^2-n+2\]
so the algorithm is $\mathcal{O}(n^3)$.

\subsubsection*{c}
Sure, for a $100\times100$ matrix, $n=100\implies n^3=1,000,000$. Given that
the bulk of the expense are matrix-matrix operations (which are indeed highly
parallelizable), I'd feel comfortable quoting the ``theoretical'' maximum FLOPS
performance of my GPU, an nVidia RTX 2070 Super at 283.2 GFLOPS for FP64.
This matrix would require $\approx \frac{100^3}{283.2*10^9}=3.53$ microseconds
to calculate.

Of course, this algorithm isn't perfectly parallelizable, but even an order of
magnitude slower would be 3.53 milliseconds...I'm probably not worried about
that unless I'm finding the determinants of LOTS of these matrices.

\newpage
\section*{Problem 3: A Plethora of Inversion Methods}
\subsection*{Problem Statement}
Suppose $\mat{A}\in\mathbb{R}^{n\times n}$ is nonsingular. Then, the inverse of
$\mat{A}$ ca be computed elementwise as
\[ \left[\mat{A}^{-1}\right]_{ij} = (-1)^{i+j}\frac{\det(\mat{B}_ij)}{\det{\mat{A}}} \]
where $\mat{B}_{ij}$ is the matrix obtained by removing $j-$th row and the
$i$-th column of $\mat{A}$ Count the number of FLOPs (only dominant terms of
$n$) required to compute the inverse of $\mat{A}$ using the above equation. For
computing determinants, assume that you are using the scheme in problem 2.
Compare it with the number of FLOPs you will take to compute an LU decomposition
followed by forward and backward substitutions to compute $\mat{A}^{-1}$.

\textit{
	\textbf{Hint}: To compute $\mat{A}^{-1}$ using LU decompositions and
	forward/backward substitutions, you can calculate the columns $x_1,\ldots,x_n$
	of $\mat{A}^{-1}$ by solving $\mat{A}x_i=\mat{e}_i$ where $\mat{e}_i\in\mathbb{R}^n$
	is a vector whose elements are all zeros except its $i$-th element that is unity.
}

\subsection*{Solution}
%------------------------------------------------------------------------------------------------------------

\newpage
\printbibliography
\end{document}
