\documentclass[11pt]{report}
\usepackage{amsmath}
\usepackage{amssymb}
\usepackage{amsfonts}
\usepackage{mathtools}
\usepackage{amsthm}
\usepackage{ragged2e}
\usepackage[hidelinks]{hyperref}
\usepackage{float}
\usepackage{pgf,tikz}
\usepackage[shortlabels]{enumitem}
\usepackage{color}
\usepackage{pgfplots}
\usepackage[margin = 1 in]{geometry}
\usepackage{mathrsfs}
\usetikzlibrary{arrows}
\usepackage{multicol}
\usepackage{fancyhdr}
\pagestyle{fancy}
\usepackage{multirow}
\usepackage{graphicx}
\usepackage{psfrag}
\usepackage{biblatex}
\addbibresource{hw6.bib}
\usepackage{listings}
\renewcommand{\footrulewidth}{0.4pt}

\newtheorem{theorem}{Theorem}[chapter]
\newtheorem{defn}{Definition}[chapter]
\newtheorem{lemma}{Lemma}[chapter]

\theoremstyle{definition}
\newtheorem{proposition}{Proposition}[chapter]
\newtheorem{remark}{Remark}[chapter]
\newtheorem{example}{Example}[chapter]

\DeclareMathOperator*{\argmin}{arg\,min}
\DeclareMathOperator*{\argmax}{arg\,max}

\newcommand{\user}{}
\newcommand{\xlr}[2]{#1 \left(#2\right)}
\newcommand{\clr}[2]{#1 \left\{ #2 \right\}}
\newcommand{\rank}{\mathrm{rank}}
\newcommand{\mat}[1]{\mathbf{#1}}
\lhead{ECE 530 - Fall 2023 at University of Illinois at Urbana-Champaign}
\rhead{HW5}
\lfoot{Author: \textcolor{red}{Eric Silk, esilk2}}
\rfoot{Due: Friday, Oct. 20th}
\begin{document}


\section*{Trapezoid o' trapezoid!}
\subsection*{Problem Statement}
Consider the scalar ordinary differential equation (ODE) $\dot{x}=f(t,x(t))$
starting from $x_0=x(0)$. Then, the trajectory $x(t)$ over $t\in[0,T]$ satisfies:
\[x(t_{n+1})=x(t_n)+\int_{\tau=t_n}^{\tau=t_{n+1}}f(\tau,x(\tau))d\tau\]
for $t_n=nh$. Let $g(\tau)\coloneqq f(\tau,(x(\tau)))$. Assume that $g$ is $\mathcal{C}^2$
over $[0,T]$.

\subsubsection*{a}
If $0\leq a\leq b\leq T$, then prove that $g$ satisfies:
\[
	\int_{a}^{b} g(\tau)d\tau
	= \frac{1}{2}(b-a)[g(a)+g(b)]
	- \frac{1}{2}\int_{a}^{b}\left[
		\left(\frac{b-a}{2}\right)^2 -
		\left(\tau - \frac{a+b}{2}\right)^2
		\right]
	g''(\tau)d\tau
\]
\textbf{Hint}: Use integration by parts.
\subsubsection*{b}
Then, use the identity in part a) to prove that there exists a constant $M$ s.t.
\[
	\bigg|
	\int_{a}^{b}g(\tau)d\tau - \frac{1}{2}(b-a)[g(a)+g(b)]
	\bigg|
	\leq M\frac{(b-a)^3}{12}
\]

\subsubsection*{c}
Deduce that the local truncation error of the trapezoidal method is $\mathcal{O}(h^3)$;
i.e.:
\[
	x(t_{n+1})=x(t_n) + \frac{h}{2}[f(t_n,x(t_n))+f(t_{n+1},x(t_{n+1}))]
	+ \mathcal{O}(h^3)
\]

\subsubsection*{d (\textit{optional})}
Under possibly additional assumptions, prove that the global error for the trapezoidal
method is $\mathcal{O}(h^2)$; i.e., there exists some constant $C$ s.t.
\[|x(t_n)-x_n|\leq Ch^2\forall t\in[0, T]\]


\subsection*{Solution}
\subsubsection*{a}
Consider:
\[
	\int_{a}^{b}\left[\left(\frac{b-a}{2}\right)^2-\left(\tau-\frac{a+b}{2}\right)^2\right]g''(\tau)d\tau
\]
Letting:
\[
	u=\left[\left(\frac{b-a}{2}\right)^2-\left(\tau-\frac{a+b}{2}\right)^2\right]
	\implies du = a+b-2\tau d\tau
\]
\[
	dv=g''(\tau)d\tau \implies v = g'(\tau)
\]
\[
	\int_{a}^{b}udv = uv|_a^b-\int_{a}^{b}vdu \equiv
	\left[\left(\left(\frac{b-a}{2}\right)^2-\left(\tau-\frac{a+b}{2}\right)^2\right)
		g'(\tau)
		\right]_a^b
	-\int_{a}^{b}g'(\tau)(a+b-2\tau)d\tau
\]
The whole left term evaluated at the end points reduces to $0$, so:
\[ = -\int_{a}^{b}g'(\tau)(a+b-2\tau)d\tau \]
IBP again using:
\[u = a+b-2\tau \implies du = -2d\tau\]
\[dv = g'(\tau)d\tau \implies v=g(\tau)\]
\[ = (a+b-2b)g(b)-\left[(a+b-2a)g(a)\right]+2\int_{a}^{b}g(\tau)d\tau \]
\[ = (a-b)(g(a)+g(b)) + 2\int_{a}^{b}g(\tau)d\tau\]
Hokay. Now we substitute back into the original equation...
\[
	\int_{a}^{b}g(\tau)d\tau = \frac{1}{2}(b-a)(g(a)+g(b))-\frac{1}{2}
	\left[ (a-b)(g(a)+g(b)) + 2\int_{a}^{b}g(\tau)d\tau \right]
\]
\[
	\int_{a}^{b}g(\tau)d\tau = \int_a^bg(\tau)d\tau
\]
\qed

\textcolor{red}{TODO come back and figure out the sign shenanigans}

\subsubsection*{b}
Based on the proof provided by \cite{Heck_Schut}.
Consider a partition of the Trapezoidal rule:
\[x_k=a+kh,\ k=0,1,\ldots,n,\ h=\frac{b-a}{n}\]
And the full trapezoidal rule is given as:
\[T = \frac{h}{2} (f(a)+f(b))+h\sum_{k=1}^{n-1}f(a+kh)\]
Consider the sub-interval $[x_{k-1}, x_k]$ for $k=1,\ldots,n$. An estimate
of the error of this sub-interval is given by our function $g$:
\[ g(x) = f(x)-f(x_{k-1})-\frac{(f(x_k)-f(x_{k-1}))(x-x_{k-1})}{h} \]
We have previously proven that:
\[\int_{x_{k-1}}^{x_k}g(x)dx = -\frac{1}{2}\int_{x_{k-1}}^{x_k}(x-x_{k-1})(x_k-x)g''(x)dx\]
Because $g$ is $\mathcal{C}^2$ on the finite interval, we know that it's second derivative
must be bounded. Let us say:
\[M = \max_{x\in[a,b]} |g''(x)|\]
By definition, $g''(x) = f''(x)$ so:
\[
	\bigg|\int_{x_{k-1}}^{x_k}g(x)dx\bigg| \leq
	\frac{1}{2}\int_{x_{k-1}}^{x_k}(x-x_{k-1})(x_k-x)|f''(x)|dx
\]
\[ \leq \frac{M}{2}\int_{x_{k-1}}^{x_k}(x-x_{k-1})(x_{k}-x)dx \]
\[ = \frac{M}{2}\int_{x_{k-1}}^{x_k} (-x^2+(x_{k-1}+x_k)x-x_{k-1}x_k) dx \]
\[=\frac{M}{12}(x_k-x_{k-1})^3\]
Noting that we've used $x_k-x_{k-1}$ here instead of $b-a$, we have:
\[=\frac{M}{12}(b-a)^3\]
\subsection*{c}
Note that $b-a=h$. So:
\[T=\frac{M}{12}h^3\]
which is clearly $\mathcal{O}(h^3)$.
\subsection*{d}
Using the prior work (i.e. considering the sum of all partitions for global
error), we can further say:
\[
	\bigg|\int_{a}^{b}f(x)dx - T\bigg| \leq \sum_{k=1}^{n}\frac{1}{12}Mh^3
	= \frac{1}{12}Mh^3n = \frac{1}{12}M(b-a)h^2
\]

\newpage
\section*{Problem 2: Deriving Adams-Moulton, not 1, but 2}
\subsection*{Problem Statement}
Consider a scalar ODE $\dot{x}(t)=f(t,x(t))$ with $x(0)$ as the initial point.
With a step-size of $h>0$ and $t_n=nh$ for $n\geq 0$ we have:
\[ x(t_{n+1}) = x(t_n)+\int_{\tau=t_n}^{\tau=t_{n+1}} f(\tau, x(\tau))d\tau \]

Adams-Moulton seeks to compute $x_n$ recursively as:
\begin{equation}
	x_{n+1} = x_n+\int_{\tau=t_n}^{\tau=t_{n+1}} g(\tau)d\tau
	\label{eq:first}
\end{equation}
where $g(\tau)$ is a polynomial approximation to $f(\tau,x(\tau))$. Define $G_n\coloneqq f(t_n, x_n)$.
Then, the AM(2) method seeks a quadratic approximation to $g$ that takes the values
\begin{equation}
	g(t_{n-1})=G_{n-1},\ g(t_n)=G_n,\ g(t_{n+1})=G_{n+1}
	\label{eq:second}
\end{equation}
Consider $g$ of the form
\begin{equation}
	g(\tau) = G_{n-1}L_{n-1}(\tau)+G_nL_n(\tau)+G_{n+1}L_{n+1}(\tau)
	\label{eq:third}
\end{equation}
where $L$'s are the Legendre polynomials defined by:
\[
	L_{n-1}(\tau)\coloneqq \frac{(\tau-t_n)(\tau-t_{n+1})}{(t_{n-1}-t_n)(t_{n-1}-t_{n+1})},\
	L_{n}  (\tau)\coloneqq \frac{(\tau-t_{n-1})(\tau-t_{n+1})}{(t_{n}-t_{n-1})(t_{n}-t_{n+1})},\
	L_{n+1}(\tau)\coloneqq \frac{(\tau-t_{n-1})(\tau-t_{n})}{(t_{n+1}-t_{n-1})(t_{n+1}-t_{n})}
\]

\subsubsection*{a}
Verify that $g$ defined in (\ref{eq:third}) satisfies (\ref{eq:second}).
\subsubsection*{b}
Compute $\int_{\tau=t_n}^{\tau=t_{n+1}}L_{k}(\tau)d\tau$ for $k=n-1,n,n+1$.

\subsection*{c}
Plug your results from part (b) into the relation
\[
	x_{n+1} = x_n + \sum_{k=n-1}^{n+1}G_k\int_{\tau=t_n}^{\tau=t_{n+1}}L_k(\tau)d\tau
\]
Replace $G_k$ with $f(t_k, x_k)$ for $k=n-1,n,n+1$ in the above equation to find
the implicit relation among $x_{n+1}$, $x_n$, and $x_{n-1}$ for the AM(2)
method. Compare your result to AM(2) in the class notes or Wikipedia.

\subsubsection*{d}
Design an approximate AM(2) method by utilizing one step of a method of your
choice to solve the implicit equation you derived at each iteration, starting
from the forward Euler solution.

\subsection*{Solution}
\subsubsection*{a}
Consider $g(t_{n-1})$ and note that $\tau=t_{n-1}$ causes the numerators
of $L_n(\tau)$ and $L_{n+1}(\tau)$ to go to zero. Furthermore,
note that:
\[
	L_{n_1}(t_{n-1}) =
	\frac{(t_{n-1}-t_n)(t_{n-1}-t_{n+1})}{(t_{n-1}-t_n)(t_{n-1}-t_{n+1})}
	= 1
\]
\[ \implies g(t_{n-1})=G_{n-1}\checkmark\]
Considering $g(t_n)$, we see that for $\tau=t_n$ makes the numerators of
$L_{n-1}, L_{n+1}$ go to zero. Furthermore, note that:
\[
	L_{n}  (t_n) = \frac{(t_n-t_{n-1})(t_n-t_{n+1})}{(t_{n}-t_{n-1})(t_{n}-t_{n+1})}
	= 1
\]
\[\implies g(t_n)=G_n\checkmark\]
To save on some typing, we can see that this pattern continues for $t_{n+1}$
and, indeed:
\[g(t_{n+1}) = G_{n+1}\checkmark\]

\subsubsection*{b}
Using Wolfram (using a,b, and c instead of $t_{n-1},t_n,t_{n+1}$ and then making
the appropriate replacements...), we find:
\[\int_{t_n}^{t_{n+1}}L_{n-1}(\tau)d\tau = \frac{(t_n-t_{n+1})^3}{6(t_{n-1}-t_n)(t_{n-1}-t_{n+1})}\]
\[\int_{t_n}^{t_{n+1}}L_{n}(\tau)d\tau = \frac{(t_{n}-t_{n+1})(3t_{n-1}-2t_{n}-t_{n+1})}{6(t_n-t_{n-1})}\]
\[\int_{t_n}^{t_{n+1}}L_{n+1}(\tau)d\tau = \frac{(t_{n}-t_{n+1})(3t_{n-1}-t_{n}-2t_{n+1})}{6(t_{n-1}-t_{n+1})}\]


%------------------------------------------------------------------------------------------------------------

\newpage
\printbibliography
\end{document}
