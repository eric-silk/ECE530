\documentclass[11pt]{report}
\usepackage{amsmath}
\usepackage{amssymb}
\usepackage{amsfonts}
\usepackage{mathtools}
\usepackage{amsthm}
\usepackage{ragged2e}
\usepackage[hidelinks]{hyperref}
\usepackage{float}
\usepackage{pgf,tikz}
\usepackage[shortlabels]{enumitem}
\usepackage{color}
\usepackage{pgfplots}
\usepackage[margin = 1 in]{geometry}
\usepackage{mathrsfs}
\usetikzlibrary{arrows}
\usepackage{multicol}
\usepackage{fancyhdr}
\pagestyle{fancy}
\usepackage{multirow}
\usepackage{graphicx}
\usepackage{psfrag}
\usepackage{biblatex}
\addbibresource{hw6.bib}
\usepackage{listings}
\renewcommand{\footrulewidth}{0.4pt}

\newtheorem{theorem}{Theorem}[chapter]
\newtheorem{defn}{Definition}[chapter]
\newtheorem{lemma}{Lemma}[chapter]

\theoremstyle{definition}
\newtheorem{proposition}{Proposition}[chapter]
\newtheorem{remark}{Remark}[chapter]
\newtheorem{example}{Example}[chapter]

\DeclareMathOperator*{\argmin}{arg\,min}
\DeclareMathOperator*{\argmax}{arg\,max}

\newcommand{\user}{}
\newcommand{\xlr}[2]{#1 \left(#2\right)}
\newcommand{\clr}[2]{#1 \left\{ #2 \right\}}
\newcommand{\rank}{\mathrm{rank}}
\newcommand{\mat}[1]{\mathbf{#1}}
\lhead{ECE 530 - Fall 2023 at University of Illinois at Urbana-Champaign}
\rhead{HW5}
\lfoot{Author: \textcolor{red}{Eric Silk, esilk2}}
\rfoot{Due: Friday, Oct. 20th}
\begin{document}


\section*{Trapezoid o' trapezoid!}
\subsection*{Problem Statement}
Consider the scalar ordinary differential equation (ODE) $\dot{x}=f(t,x(t))$
starting from $x_0=x(0)$. Then, the trajectory $x(t)$ over $t\in[0,T]$ satisfies:
\[x(t_{n+1})=x(t_n)+\int_{\tau=t_n}^{\tau=t_{n+1}}f(\tau,x(\tau))d\tau\]
for $t_n=nh$. Let $g(\tau)\coloneqq f(\tau,(x(\tau)))$. Assume that $g$ is $\mathcal{C}^2$
over $[0,T]$.

\subsubsection*{a}
If $0\leq a\leq b\leq T$, then prove that $g$ satisfies:
\[
	\int_{a}^{b} g(\tau)d\tau
	= \frac{1}{2}(b-a)[g(a)+g(b)]
	- \frac{1}{2}\int_{a}^{b}\left[
		\left(\frac{b-a}{2}\right)^2 -
		\left(\tau - \frac{a+b}{2}\right)^2
		\right]
	g''(\tau)d\tau
\]
\textbf{Hint}: Use integration by parts.
\subsubsection*{b}
Then, use the identity in part a) to prove that there exists a constant $M$ s.t.
\[
	\bigg|
	\int_{a}^{b}g(\tau)d\tau - \frac{1}{2}(b-a)[g(a)+g(b)]
	\bigg|
	\leq M\frac{(b-a)^3}{12}
\]

\subsubsection*{c}
Deduce that the local truncation error of the trapezoidal method is $\mathcal{O}(h^3)$;
i.e.:
\[
	x(t_{n+1})=x(t_n) + \frac{h}{2}[f(t_n,x(t_n))+f(t_{n+1},x(t_{n+1}))]
	+ \mathcal{O}(h^3)
\]

\subsubsection*{d (\textit{optional})}
Under possibly additional assumptions, prove that the global error for the trapezoidal
method is $\mathcal{O}(h^2)$; i.e., there exists some constant $C$ s.t.
\[|x(t_n)-x_n|\leq Ch^2\forall t\in[0, T]\]


\subsection*{Solution}
\subsubsection*{a}
Recall that IBP is:
\[ \int_a^b udv = uv\bigg|_a^b - \int_a^b vdu \]
Letting $u\coloneqq=g(\tau)$ and $vd\tau$, then:
\[ \int_a^bg(\tau)d\tau = g(\tau)\tau\bigg|_a^b - \int_a^b\tau g'(\tau)d\tau \]
Using IBP again for the right-most term, this time letting $u=g'(\tau)$ and $dv=\tau d\tau$:
\[
	= g(\tau)\tau\bigg|_a^b - \left(
	\left[g'(\tau)-\frac{1}{2}\tau^2\right]_a^b
	- \int_a^b\tau g''(\tau)d\tau
	\right)
\]
\[
	= g(\tau)\tau\bigg|_a^b -
	\left[g'(\tau)-\frac{1}{2}\tau^2\right]_a^b
	+ \int_a^b\tau g''(\tau)d\tau
\]



I used \cite{Bender} for reference.
\subsubsection*{b}
Re-arrange the identity to:
\[
	\int_a^bg(\tau)d\tau - \frac{1}{2}(b-a)[g(a)+g(b)] = -\int_a^b\left[
		\left(\frac{b-a}{2}\right)^2-\left(\tau - \frac{a+b}{2}\right)^2
		\right] g''(\tau)d\tau
\]
Note that, because $g(\tau)$ is $\mathcal{C}^2$ over $[0,T]$ (a bounded domain),
this implies that $|g''(\tau)|$ is bounded.
Using the power of foresight, let us define $M$ as follows:
\[M = 2g''\left(\argmax_{\tau\in[0,T]}|g''(\tau)|\right)\]
Or, in English, the most ``extreme'' value that $g''$ can take (rather than the
maximum or minimum). Then:
$\int_{}^{max}$

\subsubsection*{c}
\subsubsection*{d}
Globally, the error will be the sum of each local interval; i.e.:
\[
	\int_{a}^{b}g(\tau)d\tau = \int_{a_0}^{a_1}g(\tau)d\tau + \ldots + \int_{a_{n}}^{b}g(\tau)d\tau
\]
The error on a subinterval
\[\int_{a_{i}}^{a_{i+1}} g(\tau)d\tau= -\frac{a_{i+1}-a_{i}}{12}g''(\tau)\]

\newpage
\section*{Problem 2: Deriving Adams-Moulton, not 1, but 2}
\subsection*{Problem Statement}
Consider a scalar ODE $\dot{x}(t)=f(t,x(t))$ with $x(0)$ as the initial point.
With a step-size of $h>0$ and $t_n=nh$ for $n\geq 0$ we have:
\[ x(t_{n+1}) = x(t_n)+\int_{\tau=t_n}^{\tau=t_{n+1}} f(\tau, x(\tau))d\tau \]

Adams-Moulton seeks to compute $x_n$ recursively as:
\begin{equation}
	x_{n+1} = x_n+\int_{\tau=t_n}^{\tau=t_{n+1}} g(\tau)d\tau
	\label{eq:first}
\end{equation}
where $g(\tau)$ is a polynomial approximation to $f(\tau,x(\tau))$. Define $G_n\coloneqq f(t_n, x_n)$.
Then, the AM(2) method seeks a quadratic approximation to $g$ that takes the values
\begin{equation}
	g(t_{n-1})=G_{n-1},\ g(t_n)=G_n,\ g(t_{n+1})=G_{n+1}
	\label{eq:second}
\end{equation}
Consider $g$ of the form
\begin{equation}
	g(\tau) = G_{n-1}L_{n-1}(\tau)+G_nL_n(\tau)+G_{n+1}L_{n+1}(\tau)
	\label{eq:third}
\end{equation}
where $L$'s are the Legendre polynomials defined by:
\[
	L_{n-1}(\tau)\coloneqq \frac{(\tau-t_n)(\tau-t_{n+1})}{(t_{n-1}-t_n)(t_{n-1}-t_{n+1})},\
	L_{n}  (\tau)\coloneqq \frac{(\tau-t_{n-1})(\tau-t_{n+1})}{(t_{n}-t_{n-1})(t_{n}-t_{n+1})},\
	L_{n+1}(\tau)\coloneqq \frac{(\tau-t_{n-1})(\tau-t_{n})}{(t_{n+1}-t_{n-1})(t_{n+1}-t_{n})}
\]

\subsubsection*{a}
Verify that $g$ defined in (\ref{eq:third}) satisfies (\ref{eq:second}).
\subsubsection*{b}
Compute $\int_{\tau=t_n}^{\tau=t_{n+1}}L_{k}(\tau)d\tau$ for $k=n-1,n,n+1$.

\subsection*{c}
Plug your results from part (b) into the relation
\[
	x_{n+1} = x_n + \sum_{k=n-1}^{n+1}G_k\int_{\tau=t_n}^{\tau=t_{n+1}}L_k(\tau)d\tau
\]
Replace $G_k$ is a polynomial approximation to $f(\tau, x(\tau))$.

\subsection*{Solution}


%------------------------------------------------------------------------------------------------------------

\newpage
\printbibliography
\end{document}
